X\documentstyle[11pt] {report}
X\marginparwidth 0pt
X\oddsidemargin  0pt
X\evensidemargin  0pt
X\marginparsep 0pt
X\topmargin=0.0in
X\textwidth=6.25in
X\textheight=8.25in
X\pagestyle{plain}
X\parskip=5pt
X\parindent=30pt
X
X\begin{document}
X\thispagestyle{empty}
X
X\centerline{\Huge\bf  ASURV}
X\bigskip
X\centerline{\Huge\bf Astronomy SURVival Analysis }
X
X\bigskip
X\bigskip
X\bigskip
X\bigskip
X\centerline{\LARGE\bf  A Software Package for Statistical Analysis of }
X
X\bigskip
X\centerline{\LARGE\bf  Astronomical Data Containing Nondetections}
X
X\bigskip
X\bigskip
X\bigskip
X\bigskip
X\begin{tabbing}
X      xxxxxxxxxxx\= \kill
X                 \>xxxxxxxxxxxxxxx\=   \kill
X
X      {\Large\bf Developed by: } \\
X    \> \\
X                 \> {\Large\bf Takashi Isobe } (Center for Space Research, MIT)  \\
X
X    \> \\
X           \> {\Large\bf Michael LaValley } (Dept. of Statistics, Penn State)\\
X
X    \> \\
X                 \>{\Large\bf Eric Feigelson  }(Dept. of Astronomy \& Astrophysics, Penn State) \\
X    \> \\
X    \> \\
X    \> \\
X
X
X       {\Large\bf Available from: }   \\
X                 \>                   \\
X                 \> code@stat.psu.edu \\
X                 \>                   \\
X                 \>or,                \\
X                 \>                   \\
X                 \> Eric Feigelson     \\
X                 \> Dept. of Astronomy \& Astrophysics  \\
X                 \> Pennsylvania State University \\
X                 \> University Park PA  16802 \\
X                 \> (814) 865-0162 \\
X                 \>  Email:  edf@astro.psu.edu  (Internet)\\
X\end{tabbing}
X\bigskip
X\bigskip
X\bigskip
X\bigskip
X\bigskip
X\centerline{\Large\bf Rev. 1.2, Summer 1992.}
X\newpage
X
X
X\centerline{\Large\bf TABLE OF CONTENTS}
X\bigskip
X\bigskip
X			   
X\noindent{\large\bf 
X\begin{center}
X\begin{verbatim}
X       1 Introduction ............................................  3
X
X       2 Overview of ASURV .......................................  4 
X            2.1  Statistical Functions and Organization ..........  4 
X            2.2  Cautions and caveats ............................  6
X       
X       3 How to run ASURV ........................................  9
X            3.1  Data Input Formats ..............................  9
X            3.2  KMESTM instructions and example ................. 10 
X            3.3  TWOST instructions and example .................. 11 
X            3.4  BIVAR instructions and example .................. 13 
X       
X       4 Acknowledgements ........................................ 21
X       
X       Appendices ................................................ 22
X            A1  Overview of survival analysis .................... 22 
X            A2  Annotated Bibliography on Survival Analysis ...... 23 
X            A3  How Rev 1.2 is Different From Rev 0.0 ............ 26
X            A4  Errors Removed in Rev 1.1 ........................ 28
X            A5  Errors Removed in Rev 1.2 ........................ 28
X            A6  Obtaining and Installing ASURV ................... 28
X            A7  User Adjustable Parameters ....................... 30
X            A8  List of subroutines used in ASURV Rev 1.2 ........ 32
X\end{verbatim}
X\end{center}
X}
X
X
X\bigskip
X
X\bigskip
X
X\bigskip
X
X\centerline{\Large\bf NOTE}
X
X\begin{small}
X
XThis program and its documentation are provided `AS IS' without
Xwarranty of any kind.  The entire risk as the results and performance
Xof the program is assumed by the user.  Should the program prove
Xdefective, the user assume the entire cost of all necessary correction.
XNeither the Pennsylvania State University nor the authors of the program
Xwarrant, guarantee or make any representations regarding use of, or the
Xresults of the use of the program in terms of correctness, accuracy 
Xreliability, or otherwise.  Neither shall they be liable for any direct or
Xindirect damages arising out of the use, results of use or inability to
Xuse this product, even if the University has been advised of the possibility
Xof such damages or claim.  The use of any registered Trademark depicted
Xin this work is mere ancillary; the authors have no affiliation with
Xor approval from these Trademark owners.
X
X\end{small}
X
X\newpage
X
X\centerline{\Large\bf 1  Introduction} 
X
X    Observational astronomers frequently encounter the situation where they 
Xobserve a particular property (e.g. far infrared emission, calcium line 
Xabsorption, CO emission) of a previously defined sample of objects, but 
Xfail to detect all of the objects.  The data set then contains nondetections as 
Xwell as detections, preventing the use of simple and familiar statistical 
Xtechniques in the analysis.   
X 
X     A number of astronomers have recently recognized the existence 
Xof statistical methods, or have derived similar methods, to deal with these 
Xproblems.  The methods are collectively
Xcalled `survival analysis' and nondetections are called 
X`censored' data points.  {\sl\bf ASURV} is a menu-driven stand-alone computer
Xpackage designed  to assist astronomers in using methods from survival 
Xanalysis.  Rev. 1.2 of {\sl\bf ASURV}  provides the maximum-likelihood
Xestimator of the censored distribution,
Xseveral two-sample tests, correlation tests and linear regressions  as
Xdescribed in our papers in the {\it\bf Astrophysical Journal} (Feigelson 
Xand Nelson, 1985; Isobe, Feigelson, and Nelson, 1986). 
X
X     No statistical procedure can magically recover information that was
Xnever measured at the telescope.  However, there is frequently important 
Xinformation implicit in the failure to detect some objects which can be
Xpartially recovered under reasonable assumptions. We purposely provide 
Xa variety of statistical tests - each 
Xbased on different models of where  upper limits truly lie - so that the
Xastronomer can judge the importance of the different assumptions.  
XThere are also reasons for {\bf not} applying these methods.  We describe
Xsome of their known limitations in {\bf \S  2.2}.
X
X     Because astronomers are frequently unfamiliar with the field of 
Xstatistics, we devote Appendix {\bf  \S A1} to background 
Xmaterial.  Both general issues concerning censored data and specifics 
Xregarding the methods used in {\sl\bf ASURV} are discussed.  More 
Xmathematical presentations can be found in the references given in Appendix
X{\bf \S A2}. Users of Rev 0.0, distributed between 1988 and 1991,  are 
Xstrongly encouraged to 
Xread Appendices {\bf \S A3-A5} to be familiar with the changes made in the 
Xpackage.  Appendices {\bf \S A6-A8} are needed only by code installers and
Xthose needing to modify the I/O or array sizes.  
XUsers wishing to get right down to work should read  {\bf \S 2.1} to
Xfind the appropriate program, and follow  the instructions in {\bf \S 3}. 
X
X\bigskip
X\bigskip
X\centerline{\Large\bf Acknowledging ASURV}
X
X     We would appreciate that studies using this package include phrasing
Xsimilar to `... using {\sl\bf ASURV} Rev 1.2 ({\bf B.A.A.S.} reference),
Xwhich implements the methods presented in ({\bf Ap. J.} reference)', where
Xthe {\bf B.A.A.S.} reference is the most recent published {\sl\bf ASURV}
XSoftware Report (Isobe and Feigelson 1990; LaValley, Isobe and Feigelson 
X1992) and the {\bf Ap. J.} reference is Feigelson and Nelson (1985) for
Xunivariate problems and Isobe,Feigelson and Nelson (1986) for bivariate
Xproblems.  Other general works appropriate for referral include the review
Xof survival methods for astronomers by Schmitt (1985), and the survival 
Xanalysis monographs by Miller (1981) or Lawless (1982).
X
X\newpage
X\centerline{\Large\bf 2 Overview of ASURV}
X    
X       
X\medskip
X\noindent{\large\bf 2.1 Statistical Functions and Organization}
X
X     The statistical methods for dealing with censored data might be
Xdivided into a 2x2 grid: parametric $vs.$ nonparametric, and univariate $vs.$
Xbivariate.  Parametric methods assume that the censored survival times
Xare drawn from a parent population with a known distribution function ($e.g.$
XGaussian, exponential), and this is the principal assumption of survival
Xmodels for industrial reliability.  Nonparametric models make 
Xno assumption about the 
Xparent population, and frequently rely on maximum-likelihood techniques. 
XUnivariate methods are devoted to determining the characteristics of the
Xpopulation from which the censored sample is drawn, and comparing such 
Xcharacteristics for two or more censored samples.  Bivariate methods
Xmeasure whether the censored property of the sample is correlated with 
Xanother property of the objects and, if so, to perform a regression  that 
Xquantifies the relationship between the two variables.
X 
X     We have chosen to concentrate on nonparametric models, since  the 
Xunderlying distribution of astronomical populations is usually unknown. 
XThe linear regression methods however, are either fully parametric 
X($e.g.$ EM algorithm regression) or semi-parametric
X($e.g.$ Cox regression, Buckley-James regression). 
XMost of the methods are described in more detail in the astronomical 
Xliterature by Schmitt (1985), Feigelson and Nelson (1985) and Isobe et al. 
X(1986).  The generalized Spearman's rho utilized in {\sl\bf ASURV} 
XRev 1.2 is derived by Akritas (1990).
X
X     The program within {\sl\bf ASURV} giving univariate methods is 
X{\sl\bf UNIVAR}.  Its first subprogram is {\sl\bf KMESTM}, giving the 
XKaplan-Meier estimator for the distribution function of a randomly 
Xcensored sample.  First derived in 
X1958, it lies at the root of non-parametric survival analysis.  It is the
Xunique, self-consistent, generalized maximum-likelihood estimator for the
Xpopulation from which the sample was drawn. When formulated in cumulative 
Xform, it has analytic asymptotic (for large N) error bars. The median is
Xalways well-defined, though the mean is not if the lowest point in the 
Xsample is an upper limit.  It is identical to the differential 
X`redistribute-to-the-right' algorithm, independently derived by Avni et al.
X(1980) and others, but the differential form does not have simple analytic 
Xerror analysis.  
X
X     The second univariate program is {\sl\bf TWOST}, giving a variety 
Xof ways to test whether two censored samples are drawn from the same parent
Xpopulation.  They are mostly generalizations of standard tests for
Xuncensored data, such as the Wilcoxon and logrank nonparametric two-sample
Xtests.  They differ in how the censored data are weighted or `scored' in
Xcalculating the statistic.  Thus each is more sensitive to differences at
Xone end or the other of the distribution.  The Gehan and logrank tests are
Xwidely used in biometrics, while some of the others are not.  The tests 
Xoffered in Rev 1.2 differ significantly from those offered in Rev 0.0 and
Xdetails of the changes are in {\bf \S A3}.
X
X     {\sl\bf BIVAR} provides methods for bivariate data,  giving three 
Xcorrelation tests and three linear regression analyses. Cox hazard model 
X(correlation test), EM algorithm, and Buckley-James method (linear 
Xregressions) can treat several independent variables if the dependent 
Xvariable contains only one kind of censoring ($i.e.$ upper or lower limits). 
XGeneralized Kendall's tau (correlation  test), Spearman's rho 
X(correlation test), and Schmitt's binned linear regression can treat 
Xmixed censoring including censoring in the independent variable, but 
Xcan have only one independent variable.  
X
X	{\sl\bf COXREG} computes the correlation probabilities by Cox's 
Xproportional hazard model and {\sl\bf BHK} computes the generalized 
XKendall's tau.  {\sl\bf SPRMAN} computes correlation probabilities 
Xbased on a generalized version of Spearman's rank order correlation 
Xcoefficient.  {\sl\bf EM} calculates linear regression 
Xcoefficients by  the EM algorithm assuming a normal distribution of
Xresiduals, {\sl\bf BJ} calculates the Buckley-James regression with 
XKaplan-Meier residuals, and {\sl\bf TWOKM} calculates the binned 
Xtwo-dimensional Kaplan-Meier distribution and  associated linear 
Xregression coefficients derived by Schmitt (1985).  A bootstrapping 
Xprocedure provides error analysis for Schmitt's method in Rev 1.2.  The 
Xcode for EM algorithm follows that given by 
XWolynetz (1979) except minor changes. The code for Buckley-James method 
Xis adopted from Halpern (Stanford Dept. of Statistics; private 
Xcommunication).  Code for the Kaplan-Meier estimator and some of the 
Xtwo-sample tests was adapted from that given in Lee (1980).  {\sl\bf COXREG}, 
X{\sl\bf BHK}, {\sl\bf SPRMAN}, and the {\sl\bf TWOKM} code were developed 
Xentirely by us.
X   
X     Detailed remarks on specific subroutines can be found in the comments at 
Xthe beginning of each subroutine.  The reference for the source of the code 
Xfor the subroutine is given there, as well as an annotated list of the 
Xvariables used in the subroutine.
X
X\newpage
X\noindent{\Large\bf 2.2  Cautions and Caveats}
X 
X     The Kaplan-Meier estimator works with any underlying distribution
X($e.g.$ Gaussian, power law, bimodal), but only if the censoring is `random'.
XThat is, the probability that the measurement of an object is censored can
Xnot depend on the value of the censored variable.   At first glance, this
Xmay seem to be inapplicable to most astronomical problems:  we detect the
Xbrighter objects in a sample, so the distribution of upper limits always 
Xdepends on brightness.  However, two factors often serve to randomize 
Xthe censoring distribution.  First, the censored variable may not be 
Xcorrelated with the variable by which the sample was initially 
Xidentified.  Thus, infrared observations of a sample of radio bright 
Xobjects will be randomly censored if the radio and infrared emission are
Xunrelated to each other.  Second, astronomical objects in a sample usually
Xlie at different distances, so that brighter objects are not always the 
Xmost luminous.  In cases where the variable of interest is censored at 
Xa particular value ($e.g.$ flux, spectral line equivalent width, stellar 
Xrotation rate)  rather than randomly censored, then parametric methods 
Xappropriate to `Type 1' censoring should be used.  These are described by 
XLawless (1982) and Schneider (1986), but are not included in this package.
X 
X     Thus, the censoring mechanism of each study MUST be understood  
Xindividually to judge whether the censoring is or is not likely to be  
Xrandom.  A good example of this judgment process is provided by 
XMagri et al. (1988).  The appearance of the data, even if the upper limits 
Xare clustered at one end of the distribution, is NOT a reliable measure. A 
Xfrequent (if philosophically distasteful) escape from the difficulty of
Xdetermining the nature of the censoring in a given experiment is to define
Xthe population of interest to be the observed sample. The Kaplan-Meier
Xestimator then always gives a valid redistribution of the upper limits,
Xthough the result may not be applicable in wider contexts. 
X
X     The two-sample tests are somewhat less restrictive than the
XKaplan-Meier estimator, since they seek only to compare two samples rather
Xthan  determine the true underlying distribution.  Because of this, the
Xtwo-sample tests do not require that the censoring patterns of the two samples 
Xare random.  The two versions of the Gehan test in {\sl\bf ASURV} assume 
Xthat the censoring patterns of the two samples are the same, but 
Xthe version with hypergeometric variance is more reliable in case of 
Xdifferent censoring patterns.  The logrank test results appear to be 
Xcorrect as long as the censoring patterns are not very different.  
XPeto-Prentice seems to be the test least affected by differences in 
Xthe censoring patterns.  There is little known about the limitations of 
Xthe Peto-Peto test. These issues are discussed in Prentice and Marek (1979), 
XLatta (1981) and Lawless (1982). 
X
X     Two of the bivariate correlation coefficients, generalized Kendall's tau 
Xand Cox regression, are both known to be inaccurate when many tied values
Xare present in the data.  Ties are particularly common when the data is
Xbinned.  Caution should be used in these cases.  It is not known how the
Xthird correlation method, generalized Spearman's rho,  responds to ties in the
Xdata.  However, there is no reason to believe that it is more accurate than
XKendall's tau in this case, and it should also used be with caution in the 
Xpresence of tied values.
X
X     Cox regression, though widely used in biostatistical  applications,
Xformally applies only if the `hazard rate' is constant; that is,  the 
Xcumulative distribution function of the censored variable falls 
Xexponentially with increasing values.  Astronomical luminosity functions,
Xin contrast, are frequently modeled by power law distributions.  It is not 
Xclear whether or not the results of a Cox regression are significantly
Xaffected by this difference.
X
X     There are a variety of limitations to the three linear regression
Xmethods -- {\sl\bf EM}, {\sl\bf BJ}, and {\sl\bf TWOKM} -- presented here.   
XFirst, only Schmitt's binned method implemented in {\sl\bf TWOKM} works when 
Xcensoring is present in both variables.  Second, {\sl\bf\ EM} requires that 
Xthe  residuals about the fitted line follow a Gaussian distribution.  
X{\sl\bf BJ} and {\sl\bf TWOKM} are less restrictive, requiring only that the 
Xcensoring distribution about the fitted line is random.  Both we and 
XSchneider (1986) find little difference in the regression 
Xcoefficients derived from these two methods.  Third, the Buckley-James
Xmethod has a defect in that the final solution occasionally oscillates
Xrather than converges smoothly.  Fourth, there is considerable uncertainty 
Xregarding the error analysis of the regression coefficients of all three 
Xmodels.  {\sl\bf\ EM} gives analytic formulae based on asymptotic normality, 
Xwhile Avni and Tananbaum (1986) numerically calculate and examine the 
Xlikelihood surface.  BJ gives an analytic formula for the slope only, but it 
Xlies on a weak theoretical foundation.  We now provide Schmitt's bootstrap 
Xerror analysis for {\sl\bf TWOKM}, although this may be subject to high 
Xcomputational expense for large data sets.  Users may thus wish to run 
Xall methods and evaluate the uncertainty with caution.  Fifth, Schmitt's
Xbinned regression implemented in {\sl\bf TWOKM} has a number of drawbacks
Xdiscussed by Sadler et al. (1989), including slow or failed convergence
Xto the two-dimensional Kaplan-Meier distribution, arbitrary choice of
Xbin size and origin, and problems if either too many or too few bins are
Xselected.  In our judgment, the most reliable linear regression method
Xmay be the Buckley-James regression, and we suggest that Schmitt's regression
Xbe reserved for problems with censoring present in both variables. 
X
X    If we consider the field of survival analysis from a broader
Xperspective, we note a number of deficiencies with respect to censored 
Xstatistical problems in astronomy (see Feigelson, 1990).  Most importantly, 
Xsurvival analysis assumes the upper limits in a given experiment are 
Xprecisely known, while in astronomy they frequently represent n times 
Xthe RMS noise level in the experimental detector, where n = 2, 3, 5 
Xor whatever.  {\bf It is possible that all existing survival methods will
Xbe inaccurate for astronomical data sets containing many points very close
Xto the detection limit.}  Methods that combine the virtues of survival
Xanalysis (which treat censoring) and measurement error models (which
Xtreat known measurement errors in both uncensored and censored points)
Xare needed. See the discussion by Bickel and Ritov (1992) on this important
Xmatter.  A related deficiency is the absence of 
Xweighted means or regressions associated with censored samples.
XSurvival analysis also has not yet produced any true multivariate 
Xtechniques, such as a Principal Components Analysis that permits 
Xcensoring.  There also seems to be a  dearth of nonparametric 
Xgoodness-of-fit tests like the Kolmogorov-Smirnov test.
X
X    Finally, we note that this package - {\sl\bf ASURV} - is not unique. 
XNearly a dozen software packages for analyzing censored data have been 
Xdeveloped (Wagner and Meeker 1985).  Four are part of large multi-purpose 
Xcommercial statistical software systems:  SAS, SPSS, BMDP, and GLIM.  
XThese packages are available on many university central computers. We have 
Xfound BMDP to be the most useful for astronomers (see Appendices to 
XFeigelson and Nelson 1985, Isobe et al. 1986 for instructions on its use).   
XIt provides most of the functions in {\sl\bf KMESTM} and {\sl\bf TWOST} 
Xas well as a full Cox regression, but no linear regressions. Other packages 
X(CENSOR, DASH, LIMDEP, STATPAC, STAR, SURVAN, SURVCALC, SURVREG) were written 
Xat various universities, medical institutes, publishing houses and industrial 
Xlabs;  they have not been evaluated by us. 
X 
X
X\newpage
X\centerline{\Large\bf 3  How to Run ASURV}
X
X\medskip
X\noindent{\large\bf 3.1  Data Input Formats}
X 
X     {\sl\bf ASURV} is designed to be menu-driven.  There are two basic input
Xstructures:  a data file, and a command file.  The data file is assumed to
Xreside on disk.  For each observed object, it contains the measured value 
Xof the variable of interest and an indicator as to whether it is detected
Xor not.  Listed below are the possible values that the {\bf censoring indicator
X} can take on.  Upper limits are most common in astronomical applications and
Xlower limits are most common in lifetime applications. 
X
X\begin{verbatim}
X
X For univariate data:    1 : Lower Limit
X                         0 : Detection
X                        -1 : Upper Limit
X
X For bivariate data:     3 : Both Independent and Dependent Variables are 
X                               Lower Limits
X                         2 : Only independent Variable is a Lower Limit
X                         1 : Only Dependent Variable is a Lower Limit
X                         0 : Detected Point
X                        -1 : Only Dependent Variable is an Upper Limit
X                        -2 : Only Independent Variable is an Upper Limit
X                        -3 : Both Independent and Dependent Variables are
X                               Upper Limits
X
X\end{verbatim}
X
X     The command file may either reside on disk, but is more frequently
Xgiven interactively at the terminal.
X
X     For the univariate program {\sl\bf UNIVAR}, the format of the data file 
Xis determined in the subroutine {\sl\bf DATAIN}.  It is currently set 
Xat 10*(I4,F10.3) for {\sl\bf KMESTM}, where each column represents a 
Xdistinct subsample.  For {\sl\bf TWOST}, it is set at I4,10*(I4,F10.3), 
Xwhere the first column gives the sample identification number. Table 1 
Xgives an example of a {\sl\bf TWOST} data file called 'gal2.dat'. It 
Xshows a hypothetical study where six normal galaxies, six starburst galaxies 
Xand six Seyfert galaxies were observed with the IRAS satellite. The variable 
Xis the log of the 60-micron luminosity.  The problem we address are the
Xestimation of the luminosity functions of each sample, and a determination 
Xwhether they are different from each other at a statistically significant 
Xlevel.  Command file input formats are given in subroutine {\sl\bf UNIVAR}, 
Xand inputs are parsed in subroutines {\sl\bf DATA1} and {\sl\bf DATA2}.  All 
Xdata input and output formats can be changed by the user as described in 
Xappendix {\bf \S A7}.
X
X     For the bivariate program {\sl\bf BIVAR}, the data file format is 
Xdetermined by the subroutine {\sl\bf DATREG}.  It is currently set 
Xat I4,10F10.3.  In most cases, only two variables are used with input 
Xformat I4,2F10.3.  Table 2 shows an example of a bivariate censored problem. 
X Here one wishes to investigate possible relations between infrared and 
Xoptical luminosities in a sample of galaxies. {\sl\bf BIVAR} command file 
Xinput formats are sometimes a bit complex. The examples below illustrate 
Xvarious common cases.  The formats for the basic command inputs are given in 
Xsubroutine {\sl\bf BIVAR}.  Additional inputs for the Spearman's rho 
Xcorrelation, EM algorithm, Buckley-James method, and Schmitt's method are 
Xgiven in subroutines {\sl\bf R3}, {\sl\bf R4}, {\sl\bf R5}, and {\sl\bf R6} 
Xrespectively.
X 
X     The current version of {\sl\bf ASURV} is set up for data sets with 
Xfewer than 500 data points and occupies about 0.46 MBy of core memory. 
XFor problems with more than 500 points, the parameter values in the 
Xsubroutines {\sl\bf UNIVAR} and {\sl\bf BIVAR} must be changed as described 
Xin appendix {\bf\S A7}.  Note that for  the generalized Kendall's tau 
Xcorrelation coefficient, and possibly other programs,  the computation time 
Xgoes up as a high power of the number of data points. 
X
X     {\sl\bf ASURV} has been designed to work with data that can 
Xcontain either upper limits (right censoring) or lower limits (left 
Xcensoring).  Most of these procedures assume restrictions on the 
Xtype of censoring present in the data.  Kaplan-Meier estimation and 
Xthe two-sample tests presented here can work with data that has either 
Xupper limits or lower limits, but not with data containing both.  Cox 
Xregression, the EM algorithm, and Buckley-James regression can use 
Xseveral independent variables if the dependent variable 
Xcontains only one type of censored point (it can be either upper or lower 
Xlimits).  Kendall's tau, Spearman's rho, and Schmitt's binned regression can 
Xhave mixed censoring, including censoring in the independent variable, but 
Xthey can only have one independent variable.
X
X\bigskip
X\bigskip
X\noindent{\large\bf 3.2  KMESTM Instructions and Example}
X 
X     Suppose one wishes to see the Kaplan-Meier estimator for the infrared
Xluminosities of the normal galaxies in Table 1. If one runs {\sl\bf ASURV}
Xinteractively from the terminal, the conversation looks as follows:
X
X\begin{verbatim}
X
X   Data type                :  1         [Univariate data]
X   Problem                  :  1         [Kaplan-Meier]
X   Command file?            :  N         [Instructions from terminal]
X   Data file                :  gal1.dat  [up to 9 characters]
X   Title                    :  IR Luminosities of Galaxies
X                                         [up to 80 characters]
X   Number of variables      :  1         [if several variables in data file] 
X   Name of variable         :  IR        [ up to 9 characters]
X   Print K-M?               :  Y
X   Print out differential
X       form of K-M?         :  Y
X                               25.0      [Starting point is set to 25]
X                               5         [5 bins set]
X                               2.0       [Bin size is set equal to 2]
X   Print original data?     :  Y
X   Need output file?        :  Y
X   Output file              :  gal1.out  [up to 9 characters]
X   Other analysis?          :  N
X
X\end{verbatim}
XIf an output file is not specified, the results will be sent to the 
Xterminal screen.
X 
X     If a command file residing on disk is preferred, run {\sl\bf ASURV}
Xinteractively as above, specifying 'Y' or 'y' when asked if a command file is
Xavailable. The command file would then look as follows:
X\begin{verbatim}
X 
X   gal1.dat                     ... data file                      
X   IR Luminosities of Galaxies  ... problem title                         
X   1                            ... number of variables    
X   1                            ... which variable?
X   IR                           ... name of the variable        
X   1                            ... printing K-M (yes=1, no=0) 
X   1                            ... printing differential K-M (yes=1, no=0)
X   25.0                         ... starting point of differential K-M
X   5                            ... number of bins
X   2.0                          ... bin size
X   1                            ... printing data (yes=1, no=0)
X   gal1.out                     ... output file              
X
X\end{verbatim}
XAll inputs are read starting from the first column.  
X
X     The resulting output is shown in Table 3.  It shows, for example, 
Xthat the estimated normal galaxy cumulative IR luminosity  function is 0.0
Xbelow log(L) = 26.9, 0.63 $\pm$ 0.21 for 26.9 $<$  log(L) $<$ 28.5, 
X0.83 $\pm$ 0.15 for 28.5 $<$ log(L) $<$ 30.1, and 1.00 above log(L) = 30.1.
XThe estimated mean for the sample is 27.8 $\pm$ 0.5.  The 'C' beside two 
Xvalues indicates these are nondetections;  the Kaplan-Meier function does 
Xnot change at these values.
X
X\bigskip
X\bigskip
X\noindent{\large\bf 3.3  TWOST Instructions and Example}
X
X       Suppose one wishes to see two sample tests comparing the  subsamples
X  in Table 1. If one runs {\sl\bf ASURV} interactively from the terminal, the
X  conversation goes as follows:
X\begin{verbatim}  
X
X   Data type                :     1         [Univariate data]
X   Problem                  :     2         [Two sample test]
X   Command file?            :     N         [Instructions from terminal]
X   Data file                :     gal2.dat  [up to 9 characters]
X   Problem Title            :     IR Luminosities of Galaxies
X                                            [up to 80 characters]
X   Number of variables      :     1 
X                                            [if the preceeding answer is more
X                                             than 1, the number of the variable
X                                             to be tested is now requested]
X   Name of variable         :     IR        [up to 9 characters]
X   Number of groups         :     3
X   Which combination ?      :     0         [by the indicators in column one 
X                                  1          of the data set]
X   Name of subsamples       :     Normal    [up to 9 characters]
X                                  Starburst 
X   Need detail print out ?  :     N
X   Print full K-M?          :     Y 
X   Print out differential
X       form of K-M?         :     N
X   Print original data?     :     N
X   Output file?             :     Y
X   Output file name?        :     gal2.out     [up to 9 characters]
X   Other analysis?          :     N
X 
X\end{verbatim}
XA command file that performs the same operations goes as follows, after 
Xanswering 'Y' or 'y' where it asks for a command file:
X\begin{verbatim}
X 
X    gal2.dat                     ... data file                   
X    IR Luminosities of Galaxies  ... title                      
X    1                            ... number of variables      
X    1                            ... which variable?         
X    IR                           ... name of the variable   
X    3                            ... number of groups        
X    0   1   2                    ... indicators of the groups   
X    0   1   0   1                ... first group for testing 
X                                     second group for testing
X                                     need detail print out ? (if Y:1, N:0)
X                                     need full K-M print out? (if Y:1, N:0)
X    0                            ... printing differential K-M (yes=1, no=0)
X    0                            ... print original data? (if Y:1, N:0)
X    Normal                       ... name of first group    
X    Starburst                    ... name of second group        
X    gal2.out                     ... output file               
X\end{verbatim} 
X     The resulting output is shown in Table 4. It shows that the
Xprobability that the distribution of IR luminosities of normal and 
Xstarburst galaxies are similar at the 5\% level in both the Gehan and Logrank 
Xtests.
X
X\bigskip
X\bigskip
X\noindent{\large\bf 3.4  BIVAR Instructions and Example}
X
X       Suppose one wishes to test for correlation and perform regression
Xbetween the optical and infrared luminosities for the galaxy sample in
XTable 2. If one  runs {\sl\bf ASURV} interactively from the terminal, the
Xconversation looks as follow:
X\begin{verbatim} 
X
X  Data type                :  2         [Bivariate data]
X  Command file?            :  N         [Instructions from terminal]
X  Title                    :  Optical-Infrared Relation 
X                                        [up to 80 characters]
X  Data file                :  gal3.dat  [up to 9 characters]
X  Number of Indep. variable:  1
X  Which methods?           :  1         [Cox hazard model]
X    another one ?          :  Y
X                           :  4         [EM algorithm regression] 
X                           :  N
X  Name of Indep. variable  :  Optical
X  Name of Dep. variable    :  Infrared
X  Print original data?     :  N
X  Save output ?            :  Y
X  Name of Output file      :  gal3.out 
X  Tolerance level ?        :  N          [Default : 1.0E-5]
X  Initial estimate ?       :  N
X  Iteration limits ?       :  Y 
X  Max. iterations          :  50
X  Other analysis?          :  N
X\end{verbatim} 
X
X     The user may notice that the above test run includes several input
Xparameters specific to the EM algorithm (tolerance level through maximum 
Xiterations).  All of the regression procedures, particularly  Schmitt's
Xtwo-dimensional Kaplan-Meier estimator method that requires specification 
Xof the bins, require such extra inputs.  
X
X     A command file that performs the same operations goes as follows, 
Xfollowing the request for a command file name:
X\begin{verbatim} 
XOptical-Infrared Relation          ....  title     
Xgal3.dat                           ....  data file    
X1   1   2                          ....  1. number of independent variables
X                                         2. which independent variable
X                                         3. number of methods
X1   4                              ....  method numbers (Cox, and EM) 
XOptical  Infrared                  ....  name of indep.& dep
X                                         variables
X0                                  ....  no print of original data
Xgal3.out                           ....  output file name 
X1.0E-5                             ....  tolerance  
X0.0       0.0       0.0       0.0  ....  estimates of coefficients
X50                                 ....  iteration  
X\end{verbatim} 
X
X     The resulting output is shown in Table 5. It shows that the
Xcorrelation between optical and IR luminosities is significant at a
Xconfidence level P $<$ 0.01\%, and the linear fit  is 
X$log(L_{IR})\alpha(1.05 \pm 0.08)*log(L_{OPT})$. It is important to run all 
Xof the methods in order to get a more complete understanding of the 
Xuncertainties in these estimates.
X
X      If you want to use Buckley-James method, Spearman's rho, or Schmitt's 
Xmethod from a command file, you need the next extra inputs:
X
X\begin{verbatim}
X                          (for B-J method)
X1.0e-5                           tolerance
X30                               iteration
X
X                          (for Spearman's Rho)
X1                                  Print out the ranks used in computation;
X                                    if you do not want, set to 0  
X
X                          (for Schmitt's)
X10  10                             bin # for X & Y
X10                                 if you want to set the binning
X                                   information, set it to the positive
X                                   integer; if not, set to 0.
X1.e-5                              tolerance
X30                                 iteration
X0.5      0.5                       bin size for X & Y
X26.0    27.0                       origins for X & Y
X1                                  print out two dim KM estm;
X                                    if you do not need, set to 0.
X100                                # of iterations for bootstrap error 
X                                    analysis; if you do not want error 
X                                    analysis, set to 0
X-37                                negative integer seed for random number
X                                    generator used in error analysis.
X\end{verbatim}
X\newpage
X
X\centerline{\Large\bf Table 1}
X
X\bigskip
X\bigskip
X
X\centerline{\large\bf Example of UNIVAR Data Input for TWOST}
X
X\bigskip
X
X\centerline{\large\bf IR,Optical and Radio Luminosities of Normal,}
X\centerline{\large\bf  Starburst and Seyfert Galaxies}
X
X\begin{verbatim}
X                                            ____
X                        0   0      28.5       |
X                        0   0      26.9       |
X                        0  -1      29.7     Normal galaxies
X                        0  -1      28.1       |
X                        0   0      30.1       |
X                        0  -1      27.6     ____
X                        1  -1      29.0       |
X                        1   0      29.0       |
X                        1   0      30.2     Starburst galaxies
X                        1  -1      32.4       |
X                        1   0      28.5       |
X                        1   0      31.1     ____
X                        2   0      31.9       |
X                        2  -1      32.3     Seyfert galaxies
X                        2   0      30.4       |
X                        2   0      31.8     ____
X                        |   |         |
X                       #1  #2       #3
X
X                     ---I---I---------I--
X       Column #         4   8        17
X   
X    Note : #1 : indicator of subsamples (or groups)
X                If you need only K-M estimator, leave out this indicator.
X           #2 : indicator of censoring
X           #3 : variable (in this case, the values are in log form)
X\end{verbatim}
X\newpage
X
X\centerline{\Large\bf Table 2}
X
X\bigskip
X\bigskip
X
X\centerline{\large\bf Example of Bivariate Data Input for MULVAR}
X
X\bigskip
X
X\centerline{\large\bf Optical and IR luminosity relation of IRAS galaxies}
X
X\begin{verbatim}
X
X                      0      27.2      28.5 
X                      0      25.4      26.9
X                     -1      27.2      29.7
X                     -1      25.9      28.1  
X                      0      28.8      30.1 
X                     -1      25.3      27.6
X                     -1      26.5      29.0   
X                      0      27.1      29.0  
X                      0      28.9      30.2 
X                     -1      29.9      32.4
X                      0      27.0      28.5
X                      0      29.8      31.1 
X                      0      30.1      31.9   
X                     -1      29.7      32.3  
X                      0      28.4      30.4 
X                      0      29.3      31.8
X                      |       |         |
X                     #1      #2        #3
X                           _________  ______
X                           Optical      IR
X
X                   ---I---------I---------I--
X       Column #       4        13        22
X
X    Note : #1 :  indicator of censoring
X           #2 :  independent variable (may be more Than one)
X           #3 :  dependent variable
X\end{verbatim}
X\newpage
X
X\centerline{\Large\bf Table 3}
X
X\bigskip
X
X\centerline{\large\bf Example of KMESTM Output}
X
X\bigskip
X
X\begin{small}
X\begin{verbatim}
X        KAPLAN-MEIER ESTIMATOR
X    
X        TITLE : IR Luminosities of Galaxies                                                     
X        DATA FILE :  gal1.dat
X    
X        VARIABLE : IR       
X    
X        # OF DATA POINTS :   6 # OF UPPER LIMITS :   3
X    
X          VARIABLE RANGE      KM ESTIMATOR   ERROR
XFROM    0.000   TO   26.900       1.000
XFROM   26.900   TO   28.500       0.375       0.213
X       27.600 C 
X       28.100 C 
XFROM   28.500   TO   30.100       0.167       0.152
X       29.700 C 
XFROM   30.100   ONWARDS           0.000       0.000
X
X      SINCE THERE ARE LESS THAN 4 UNCENSORED POINTS,
X      NO PERCENTILES WERE COMPUTED.
X    
X        MEAN=    27.767 +/- 0.515
X  
X        DIFFERENTIAL KM ESTIMATOR
X        BIN CENTER          D
X  
X         26.000          3.750
X         28.000          1.250
X         30.000          1.000
X         32.000          0.000
X         34.000          0.000
X                       -------
X          SUM =          6.000
X  
X (D GIVES THE ESTIMATED DATA POINTS IN EACH BIN)
X        INPUT DATA FILE: gal1.dat 
X        CENSORSHIP     X 
X               0    28.500
X               0    26.900
X              -1    29.700
X              -1    28.100
X               0    30.100
X              -1    27.600
X
X\end{verbatim}
X\end{small}
X\newpage
X
X
X\centerline{\Large\bf Table 4}
X
X\bigskip
X
X\centerline{\large\bf Example of TWOST Output}
X
X\bigskip
X
X\begin{small}
X\begin{verbatim} 
X           ******* TWO SAMPLE TEST ******
X
X        TITLE : IR Luminosities of Galaxies                                                     
X        DATA SET : gal2.dat 
X        VARIABLE : IR       
X        TESTED GROUPS ARE Normal    AND Starburst
X     
X      # OF DATA POINTS :  12, # OF UPPER LIMITS :   5
X      # OF DATA POINTS IN GROUP I  :   6
X      # OF DATA POINTS IN GROUP II :   6
X     
X        GEHAN`S GENERALIZED WILCOXON TEST -- PERMUTATION VARIANCE
X     
X          TEST STATISTIC        =       1.652
X          PROBABILITY           =       0.0986
X
X     
X        GEHAN`S GENERALIZED WILCOXON TEST -- HYPERGEOMETRIC VARIANCE
X     
X          TEST STATISTIC        =       1.687
X          PROBABILITY           =       0.0917
X
X     
X        LOGRANK TEST 
X     
X          TEST STATISTIC        =       1.814
X          PROBABILITY           =       0.0696
X
X     
X        PETO & PETO GENERALIZED WILCOXON TEST
X     
X          TEST STATISTIC        =       1.730
X          PROBABILITY           =       0.0837
X
X     
X        PETO & PRENTICE GENERALIZED WILCOXON TEST
X     
X          TEST STATISTIC        =       1.706
X          PROBABILITY           =       0.0881
X
X    
X    
X        KAPLAN-MEIER ESTIMATOR
X    
X        DATA FILE :  gal2.dat
X    
X        VARIABLE : Normal   
X    
X        # OF DATA POINTS :   6 # OF UPPER LIMITS :   3
X    
X          VARIABLE RANGE      KM ESTIMATOR   ERROR
XFROM    0.000   TO   26.900       1.000
XFROM   26.900   TO   28.500       0.375       0.213
X       27.600 C 
X       28.100 C 
XFROM   28.500   TO   30.100       0.167       0.152
X       29.700 C 
XFROM   30.100   ONWARDS           0.000       0.000
X    
X
X      SINCE THERE ARE LESS THAN 4 UNCENSORED POINTS,
X      NO PERCENTILES WERE COMPUTED.
X    
X        MEAN=    27.767 +/- 0.515
X    
X    
X        KAPLAN-MEIER ESTIMATOR
X    
X        DATA FILE :  gal2.dat
X    
X        VARIABLE : Starburst
X    
X        # OF DATA POINTS :   6 # OF UPPER LIMITS :   2
X    
X          VARIABLE RANGE      KM ESTIMATOR   ERROR
XFROM    0.000   TO   28.500       1.000
XFROM   28.500   TO   29.000       0.600       0.219
X       29.000 C 
XFROM   29.000   TO   30.200       0.400       0.219
XFROM   30.200   TO   31.100       0.200       0.179
XFROM   31.100   ONWARDS           0.000       0.000
X       32.400 C 
X    
X        PERCENTILES    
X         75 TH     50 TH     25 TH
X         17.812    28.750    29.900
X    
X        MEAN=    29.460 +/- 0.460
X\end{verbatim}
X\end{small}
X\newpage
X
X
X\centerline{\Large\bf Table 5}
X
X\bigskip 
X\bigskip 
X
X\centerline{\large\bf Example of BIVAR Output}
X
X\bigskip
X
X\begin{center}
X\begin{verbatim}
X
X     
X      CORRELATION AND REGRESSION PROBLEM
X      TITLE IS  Optical-Infrared Relation                                                       
X     
X      DATA FILE IS gal3.dat 
X     
X     
X      INDEPENDENT       DEPENDENT
X        Optical   AND    Infrared
X     
X     
X      NUMBER OF DATA POINTS :    16
X      UPPER LIMITS IN  Y    X    BOTH   MIX
X                       6    0       0    0
X     
X    
X     CORRELATION TEST BY COX PROPORTIONAL HAZARD MODEL
X    
X       GLOBAL CHI SQUARE =   18.458 WITH 
X               1 DEGREES OF FREEDOM
X       PROBABILITY    =    0.0000
X    
X          
X    LINEAR REGRESSION BY PARAMETRIC EM ALGORITHM
X          
X       INTERCEPT COEFF    :  0.1703  +/-  2.2356
X       SLOPE COEFF 1      :  1.0519  +/-  0.0793
X       STANDARD DEVIATION :  0.3687
X       ITERATIONS         : 27
X          
X
X
X\end{verbatim}
X\end{center}
X 
X\newpage
X
X\centerline{\Large\bf 4  Acknowledgements}
X 
X     The production and distribution of {\sl\bf ASURV Rev 1.2} was 
Xprincipally funded at Penn State by 
Xa grant from the Center for Excellence in Space Data and Information
XSciences (operated by the Universities Space Research Association in
Xcooperation with NASA), NASA grants NAGW-1917 and NAGW-2120, and
XNSF grant DMS-9007717. T.I. was supported by NASA grant NAS8-37716.
XWe are grateful to Prof. Michael Akritas (Dept. of Statistics, Penn 
XState) for advice and guidance on mathematical issues, and
Xto Drs. Mark Wardle (Dept. of Physics and Astronomy, Northwestern
XUniversity), Paul Eskridge (Harvard-Smithsonian Center for Astrophysics),
XEric Smith (Laboratory for Astronomy and Solar Physics, NASA-Goddard
XSpace Flight Center) and Eric Jensen (Wisconsin)
Xfor generous consultation and assistance on the coding.
XWe would also like to thank Drs. Peter Allan (Rutherford Appleton Laboratory),
XSimon Morris (Carnegie Observatories), Karen Strom (UMASS), and Marco
XLolli (Bologna) for their help in debugging {\sl\bf ASURV Rev 1.0}.
X
X\newpage
X
X\bigskip
X\centerline{\Large\bf APPENDICES}
X
X\bigskip
X\bigskip
X\noindent{\large\bf A1  Overview of Survival Analysis }
X  
X     Statistical methods dealing with censored data have a long and
Xconfusing history.  It began in the 17th century with the development of
Xactuarial mathematics for the life insurance industry and demographic 
Xstudies.  Astronomer Edmund Halley was one of the founders. In the
Xmid-20th century, this application grew along with biomedical and clinical
Xresearch into a major field of biometrics. Similar (and sometimes 
Xidentical) statistical methods were  being developed in two other fields:
Xthe theory of reliability, with industrial and engineering applications; 
Xand econometrics, with attempts to understand the relations between
Xeconomic forces in groups of people. Finally, we find the same mathematical
Xproblems and some of the same solutions arising in modern astronomy as
Xoutlined in {\bf \S 1} above.
X 
X     Let us give an illustration on the use of survival analysis in these
Xdisparate fields.  Consider a linear regression problem.   First, an 
Xepidemiologist wishes to determine how the human longevity or `survival time' 
X(dependent variable) depends on the number of cigarettes smoked per day 
X(independent variable).  The experiment lasts 10 years, during which some 
Xindividuals die  and others do not. The survival time of the living 
Xindividuals is only known to be greater than their age when the experiment
Xends. These values are therefore `right censored data points'.  Second, 
Xan engine manufacturing company engines wishes to know the average time 
Xbetween breakdowns as a function of engine speed to determine the optimal
Xoperating range.  Test engines are set running until 20\% of them fail; 
Xthe average `lifetime' and dependence on speed is then calculated with 
X80\% of the data points right-censored.  Third, an astronomer observes a sample
Xof nearby galaxies in the far-infrared to determine the relation between 
Xdust and molecular gas.  Half have detected infrared luminosities  but 
Xhalf are upper limits (left-censored data points).  The astronomer then seeks
Xthe relationship between infrared luminosities and the CO traces of molecular
Xmaterial to investigate star formation efficiency.  The CO observations may 
Xalso contain censored values.
X 
X     Astronomers have dealt with their upper limits in a number of
Xfashions.  One is to consider only detections in the analysis; while
Xpossibly acceptable for some purposes (e.g. correlation),  this will 
Xclearly bias the results in others (e.g. estimating  luminosity functions).
XA second procedure considers the ratio of detected objects to observed 
Xobjects in a given sample. When plotted in a range of luminosity bins, 
Xthis has been called the `fractional luminosity function' and has been 
Xfrequently used in extragalactic radio astronomy.  This is mathematically 
Xthe same as actuarial life tables.  But it is sometimes used incorrectly 
Xin comparing different samples:  the detected fraction clearly depends on 
Xthe (usually uninteresting) distance to the objects as well as their
X(interesting)  luminosity.  Also, simple $sqrt$N error bars do not
Xapply in these fractional luminosity functions, as is frequently assumed. 
X 
X     A third procedure is to take all of the data, including the exact
Xvalues of the upper limits, and model the properties of   the parent
Xpopulation under certain mathematical constraints, such as maximizing 
Xthe likelihood that the censored points follow the same distribution as the
Xuncensored points.  This technique, which is at the root of much of modern 
Xsurvival analysis, was independently developed by astrophysicists (Avni et al.
X1980; Avni and Tananbaum 1986) and is now commonly used in observational
Xastronomy. 
X
X     The advantage accrued in learning and using survival analysis methods 
Xdeveloped in biometrics, econometrics, actuarial and reliability
Xmathematics is the wide range of useful techniques developed in these 
Xfields. In general, astronomers can have faith in the mathematical validity
Xof survival analysis.  Issues of great social importance (e.g.
Xestablishing Social Security benefits, strategies for dealing with the 
XAIDS epidemic, MIL-STD reliability standards) are based on it. In detail,
Xhowever, astronomers must study the assumptions underlying each method and
Xbe aware that some methods at the  forefront of survival analysis that may 
Xnot be fully understood.
X 
X     \S {\bf A2} below gives a bibliography of selected works concerning 
Xsurvival analysis statistical methods.  We have listed some  recent
Xmonographs from the biometric and reliability field that we have found to
Xbe useful (Kalbfleisch and Prentice 1980; Lee 1980;  Lawless 1982; Miller
X1981; Schneider 1986), as well as one from econometrics (Amemiya 1985). 
XPapers from the astronomical literature dealing with these methods include
XAvni et al. (1980), Schmitt (1985), Feigelson and Nelson (1985), Avni and 
XTananbaum (1986), Isobe et al. (1986), and Wardle and Knapp (1986).  It is
Ximportant to recognize that the methods presented in these papers and in 
Xthis software package represent only a small part of the entire body of 
Xstatistical methods applicable to censored data.
X
X
X\bigskip
X\bigskip
X\noindent{\large\bf A2  Annotated Bibliography}
X
X\begin{description}
X\item [] Akritas, M. ``Aligned Rank Tests for Regression With Censored Data'',
X   Penn State Dept. of Statistics Technical Report, 1989. \\
X   {\it Source for ASURV's generalized Spearman's rho computation.}
X
X\item [] Amemiya, T.  {\bf Advanced Econometrics} (Harvard U. Press:Cambridge 
X   MA) 1985. \\
X   {\it Reviews econometric `Tobit' regression models including censoring.}
X
X\item [] Avni, Y., Soltan, A., Tananbaum, H. and Zamorani, G. ``A Method for 
X   Determining Luminosity Functions Incorporating Both Flux Measurements 
X   and Flux Upper Limits, with Applications to the Average X-ray to Optical 
X   Luminosity Ration for Quasars", {\bf Astrophys. J.} 235:694 1980. \\
X   {\it The discovery paper in the astronomical literature for the 
X   differential Kaplan-Meier estimator.}
X
X\item [] Avni, Y. and Tananbaum, H. ``X-ray Properties of Optically Selected 
X    QSOs",     {\bf Astrophys. J.} 305:57 1986. \\
X   {\it The discovery paper in the astronomical literature of the linear 
X    regression with censored data for the Gaussian model.}
X
X\item [] Bickel, P.J and Ritov, Y. ``Discussion:  `Censoring in 
X    Astronomical Data Due
X    to Nondetections' by Eric D. Feigelson'', in {\bf Statistical Challenges
X    in Modern Astronomy}, eds. E.D. Feigelson and G.J. Babu, (Springer-Verlag:
X    New York) 1992. \\
X    {\it A discussion of the possible inadequacies of survival analysis for
X    treating low signal-to-noise astronomical data.}
X
X\item [] Brown, B .J. Jr., Hollander, M., and Korwar, R. M. ``Nonparametric 
X    Tests of Independence for Censored Data, with Applications to Heart 
X    Transplant Studies" from {\bf Reliability and Biometry}, eds. F. Proschan 
X    and R. J. Serfling (SIAM: Philadelphia) p.327 1974.\\
X    {\it Reference for the generalized Kendall's tau correlation coefficient.}
X
X\item [] Buckley, J. and James, I. ``Linear Regression with Censored Data", 
X    {\bf Biometrika} 66:429 1979.\\
X    {\it Reference for the linear regression with Kaplan-Meier residuals.}
X
X\item [] Feigelson, E. D. ``Censored Data in Astronomy'', {\bf Errors,
X    Bias and Uncertainties in Astronomy}, eds. C. Jaschek and F. Murtagh,
X    (Cambridge U. Press: Cambridge) p. 213 1990.\\
X    {\it A recent overview of the field.}
X
X\item [] Feigelson, E. D. and Nelson, P. I. ``Statistical Methods for 
X    Astronomical Data with Upper Limits: I. Univariate Distributions", 
X    {\bf Astrophys. J.} 293:192 1985.\\
X    {\it Our first paper, presenting the procedures in UNIVAR here.}
X
X\item [] Isobe, T., Feigelson, E. D., and Nelson, P. I. ``Statistical Methods 
X    for Astronomical Data with Upper Limits: II. Correlation and Regression",
X    {\bf Astrophys. J.} 306:490 1986.\\
X    {\it Our second paper, presenting the procedures in BIVAR here.}
X
X\item [] Isobe, T. and Feigelson, E. D. ``Survival Analysis, or What To Do with
X    Upper Limits in Astronomical Surveys", {\bf Bull. Inform. Centre Donnees
X    Stellaires}, 31:209 1986.\\
X    {\it A precis of the above two papers in the Newsletter of Working Group 
X    for Modern Astronomical Methodology.}
X
X\item [] Isobe, T. and Feigelson, E. D. ``ASURV'', {\bf Bull. Amer. Astro.
X    Society}, 22:917 1990.\\
X    {\it The initial software report for ASURV Rev 1.0.}
X
X\item [] Kalbfleisch, J. D. and Prentice, R. L. {\bf The Statistical Analysis 
X    of Failure Time Data} (Wiley:New York) 1980.\\
X    {\it A well-known monograph with particular emphasis on Cox regression.}
X
X\item [] Latta, R. ``A Monte Carlo Study of Some Two-Sample Rank Tests With
X    Censored Data'', {\bf Jour. Amer. Stat. Assn.}, 76:713 1981. \\
X    {\it A simulation study comparing several two-sample tests available in
X    ASURV.}
X
X\item [] LaValley, M., Isobe, T. and Feigelson, E.D. ``ASURV'', {\bf Bull.
X    Amer. Astro. Society} 1992
X    {\it The new software report for ASURV Rev. 1.1.}
X
X\item [] Lawless, J. F. {\bf Statistical Models and Methods for Lifetime Data}
X    (Wiley:New York) 1982.\\
X    {\it A very thorough monograph, emphasizing parametric models and
X    engineering applications.}
X
X\item [] Lee, E. T. {\bf Statistical Methods for Survival Data Analysis}
X    (Lifetime Learning Pub.:Belmont CA) 1980.\\
X    {\it Hands-on approach, with many useful examples and Fortran programs.}
X
X\item [] Magri, C., Haynes, M., Forman, W., Jones, C. and Giovanelli, R.
X    ``The Pattern of Gas Deficiency in Cluster Spirals: The Correlation of
X    H I and X-ray Properties'', {\bf Astrophys. J.} 333:136 1988. \\
X    {\it A use of bivariate survival analysis in astronomy, with a
X    good discussion of the applicability of the methods.}
X
X\item [] Millard, S. and Deverel, S. ``Nonparametric Statistical Methods for
X    Comparing Two Sites Based on Data With Multiple Nondetect Limits'',
X    {\bf Water Resources Research}, 24:2087 1988. \\
X    {\it A good introduction to the two-sample tests used in ASURV, written
X    in nontechnical language.}
X
X\item [] Miller, R. G. Jr. {\bf Survival Analysis} (Wiley, New York) 1981.\\
X    {\it A good introduction to the field; brief and informative lecture notes
X    from a graduate course at Stanford.}
X
X\item [] Prentice, R. and Marek, P. ``A Qualitative Discrepancy Between 
X    Censored Data Rank Tests'', {\bf Biometrics} 35:861 1979. \\
X    {\it A look at some of the problems with the Gehan two-sample test, using
X    data from a cancer experiment.}
X
X\item[] Sadler, E. M., Jenkins, C. R.  and Kotanyi, C. G..
X    ``Low-Luminosity Radio Sources in Early-Type Galaxies'', 
X    {\bf Mon. Not. Royal Astro. Soc.} 240:591 1989. \\
X    {\it An astronomical application of survival analysis, with 
X    useful discussion of the methods in the Appendices.}
X
X\item [] Schmitt, J. H. M. M. ``Statistical Analysis of Astronomical Data 
X    Containing Upper Bounds: General Methods and Examples Drawn from X-ray 
X    Astronomy", {\bf Astrophys. J.} 293:178 1985.\\
X    {\it Similar to our papers, a presentation of survival analysis for 
X    astronomers.  Derives {\sl\bf TWOKM} regression for censoring in both 
X    variables.}
X
X\item [] Schneider, H. {\bf Truncated and Censored Samples from Normal 
X    Populations} (Dekker: New York) 1986.\\
X    {\it Monograph specializing on Gaussian models, with a good discussion of 
X    linear regression.}
X
X\item [] Wagner, A. E. and Meeker, W. Q. Jr. ``A Survey of Statistical Software
X    for Life Data Analysis", in {\bf 1985 Proceedings of the Statistical 
X    Computing Section}, (Amer. Stat. Assn.:Washington DC), p. 441.\\
X    {\it Summarizes capabilities and gives addresses for other software
X     packages.}
X
X\item [] Wardle, M. and Knapp, G. R. ``The Statistical Distribution of the 
X    Neutral-Hydrogen Content of S0 Galaxies", {\bf Astron. J.} 91:23 1986.\\
X    {\it Discusses the differential Kaplan-Meier distribution and its error
X    analysis.}
X
X\item [] Wolynetz, M. S. ``Maximum Likelihood Estimation in a Linear Model 
X    from     Confined and Censored Normal Data", {\bf Appl. Stat.} 28:195 
X    1979.\\
X    {\it Published Fortran code for the EM algorithm linear regression.}
X
X\end{description}
X
X\bigskip
X\bigskip
X
X
X\noindent{\Large\bf A3 Rev 1.1 Modifications and Additions}
X
X     Each of the three major areas of {\sl\bf ASURV}; the KM 
Xestimator, the two-sample tests, and the bivariate methods have been 
Xupdated in going from Rev 0.0 to Rev 1.1. 
X
X\noindent{\large\bf A3.1 KMESTM}
X
X     In the Survival Analysis literature, the value of the survival function
Xat a x-value is the probability that a given observation will be at least as
Xlarge as that x-value.  As a result, the survival curve starts with a 
Xvalue of one and declines to zero as the x-values increase.  The 
XKM estimate of the survival curve should mirror this behavior by 
Xstarting at one and declining to zero as more and more of the observations
Xare passed.  In Rev 0.0, the KM estimate for right-censored (lower limits) 
Xdata was given in that form, but the KM estimate for left-censored
X(upper limits) data started at zero and increased to one.  As a result,
Xthe x-values where jumps in the KM estimate occurred were correct, and the 
Xjumps were of the correct height, but the reported survival value at most 
Xpoints were (in a strict sense) incorrect.  In Rev 1.1, this has been
Xcorrected so that the KM estimate will move in the proper direction for both
Xupper limits and lower limits data.  
X
X     A differential, or binned, Kaplan-Meier estimate has also been added to
Xthe package.  This allows the user to find the number of points falling into
Xspecified bins along the X-axis according to the Kaplan-Meier estimated
Xsurvival curve.  However, astronomers are strongly encouraged to use the
Xintegral KM for which analytic error analysis is available.  There is
X{\bf no known error analysis} for the differential KM.
X
X\noindent{\large\bf A3.2 TWOST}
X
X     In Rev 0.0 the code for two-sample tests relied heavily on code published
Xin {\bf Statistical Methods for Survival Data Analysis} by E.T. Lee.  Since the
Xpublication of this book in 1980, there have been several articles and 
Xsimulation studies done on the various two-sample tests.  Lee's book 
Xuses a permutation variance for its tests, which assumes that both 
Xgroups being considered are subject to the same censoring pattern.  Tests 
Xwith hypergeometric variance form seem to be more `robust' against differences
Xin the censoring patterns, and some statistical software packages 
X($e.g.$ SAS) have replaced permutation variances with hypergeometric variances.
XWe have also realized that Rev 0.0 presented Lee's Peto-Peto test, rather
Xthan the Peto-Prentice test described in Feigelson and Nelson (1985) and the
XRev. 0.0 {\sl\bf ASURV} manual.
X
X     In light of these developments we have modified the two-sample tests
Xcalculated in {\sl\bf ASURV}.  Rev 1.1 offers two versions of 
XGehan's test:  one with permutation variance (which will match the Gehan's 
Xtest value from Rev 0.0) and one with hypergeometric variance.  The logrank 
Xtest now uses a hypergeometric variance, but the Peto-Peto test still uses a 
Xpermutation variance.  The Cox-Mantel test has been removed as it was very
Xsimilar to the logrank test, and the Peto-Prentice test has been added.  The
XPeto-Prentice test uses an asymptotic variance form that has been shown to
Xdo very well in simulation studies (Latta, 1981).
X
X     {\sl\bf ASURV} now automatically does all the two-sample tests, 
Xinstead of asking the user to specify which tests to run.  These tests 
Xare not very time consuming and the user will do well to consider 
Xthe results of all the tests.  If the p-values differ significantly, 
Xthen the Peto-Prentice test is probably the most reliable (Prentice 
Xand Marek, 1979).  The two-sample tests all use different, but 
Xreasonable, weightings of the data points, so large discrepancies 
Xbetween the results of the tests indicates that caution should be 
Xused in drawing conclusions based on this data.
X
X\noindent{\large\bf A3.3 BIVAR}
X
X     The bivariate methods have been extended in two ways.  First,
Xstandard error estimates for the slope and intercept are offered for
XSchmitt's method of linear regression.  These error estimates are
Xbased upon the bootstrap, a statistical technique which randomly 
Xresamples the set of data points, with replacement, many times
Xand then runs Schmitt's procedure on the artificial data sets created by the
Xresampling.  Two hundred iterations is considered sufficient to get reasonable 
Xestimates of the standard error of the estimate in most situations.  
XHowever, this might be computationally intensive for a large data set.
X
X     
X     The bivariate methods have also been extended by an
Xadditional correlation routine, a generalized Spearman's rho procedure 
X(Akritas 1990).  The usual Spearman's rho correlation estimate for uncensored 
Xdata is simply the correlation between the ranks of the independent and 
Xdependent variables.  In order to extend the procedure to censored data, the
XKaplan-Meier estimate of the survival curve is used to assign ranks to the
Xobservations.  Consequently, the ranks assigned to the observations may not 
Xbe whole numbers.  Censored points are assigned half (for left-censored)
Xor twice (for right-censored) the rank that they would have had were
Xthey uncensored.  This method is based on preliminary findings and 
Xhas not been carefully scrutinized either theoretically or empirically.  It is
Xoffered in the code to serve as a less computer intensive approximation to the
XKendall's tau correlation, which becomes computationally infeasible for large 
Xdata sets (n $>$ 1000).  {\it The generalized Spearman's Rho routine is not 
Xdependable for small data sets (n $<$ 30)}.  In that situation the generalized
XKendall's tau routine should be relied upon.  It should also be noted that 
Xthe test statistics reported by {\sl\bf ASURV} for Kendall's tau and Spearman's
Xrho are not directly comparable.  The test statistic reported for Spearman's
Xrho is an estimated correlation, and the test statistic given for Kendall's
Xtau is an estimated function of the correlation.  It is the reported
Xprobabilities that should be compared.
X
X\noindent{\large\bf A3.4 Interface, Outputs and Manual}
X
X     The screen-keyboard interface has been streamlined somewhat.  For
Xexample. the user is now provided all two-sample tests without requesting 
Xthem individually.  New inputs have been introduced for the new programs
X(differential Kaplan-Meier function, the generalized Spearman's rho, and
Xerror analysis for Schmitt's regression).  The printed outputs of the
Xprograms have been clarified in several places where ambiguities were
Xreported.  For example, the Kaplan-Meier estimator now specifies whether the
Xchange occurs before or after a given value, the meaning of the correlation
Xprobabilities is now given explicitly, and warnings are printed when there
Xis good reason to suspect the results of a test are unreliable.  
XThe Users Manual has been reorganized so that material that is not actually 
Xneeded to operate the program has been located in several Appendices.  
X
X\bigskip
X 
X\bigskip
X
X\noindent{\Large\bf A4  Errors Removed in Rev 1.1}
X
X     Since {\sl\bf ASURV} was released in November 1991, several errors
Xhave been discovered in the package by users and have been reported to us.
XIn revision 1.1, all the bugs that have been reported are corrected.  We
Xhave also taken the opportunity to correct some subtle programming errors that
Xwe came across in the code.  The major errors were:
X
X\begin{itemize}
X
X  \item The command files gal1.com and gal2.com, provided in asurv.etc did not
X        match the new input formats.  This caused the output files created by
X        the command files not to match the examples in the manual.
X
X  \item The character variable containing the name of the data file was
X        not defined properly in the subroutine which prints out the results of
X        Kaplan-Meier estimation.
X
X  \item In subroutine TWOST, the variable WRK6 was wrongly specified as WWRK6.
X
X  \item Various problems with truncation and integer arithmetic were found
X        and removed from the code.
X
X  \item To help VAX/VMS users, a carriage control line was added to {\sl\bf 
X        ASURV}.  For information on this addition, look in {\bf \S A7}
X
X\end{itemize}
X
X
X\bigskip
X
X\noindent{\Large\bf A5 Errors removed in Rev 1.2}
X
X     After releasing {\sl\bf ASURV} Rev 1.1 in March 1992, we determined that
XRev. 1.1, and all previous versions, contained an inconsistency in the
Xway the Kaplan-Meier routine treated certain data sets.  The problem occurred 
Xwhen multiple upper limits were tied at the smallest data point, or when
Xmultiple lower limits were tied at the largest data point.  Since such an 
Xevent would be very unlikely in the biomedical setting, and {\sl\bf ASURV}
Xwas initially modeled on biomedical software, no contingency for such a
Xsituation was contained in the package.  However, this type of censoring
Xoccurs commonly in astronomical applications, so the package has been 
Xmodified to reflect this.  The Kaplan-Meier routine in Rev 1.2 temporarily 
Xchanges all such tied censored points to detections so they can be 
Xtreated consistently.
X
X{\it Other reported bugs:}  Two bugs are present in the univariate
Xtwo-sample tests.  First, the generalized Wilcoxon and Peto-Prentice 
Xunivariate two-sample tests may give different results of one switches 
Xthe sample identifiers when ties are present.  Second, in certain unlikely 
Xcircumstances when the user runs multiple tests without exiting ASURV, slight 
Xchanges in test results may be seen.  
X
X\bigskip
X
X\noindent{\Large\bf A6  Obtaining and Installing ASURV}
X 
X     This program is available as a stand-alone package  to any member of 
Xthe astronomical community without charge.  We provide the FORTRAN source code,
Xnot the executable files. We prefer to distribute it by electronic mail.  
XScientists with network connectivity should send their request to:
X
X\begin{center}
X\begin{verbatim}
X                               code@stat.psu.edu
X\end{verbatim}
X\end{center}
X
X\noindent specifying {\sl\bf `ASURV Rev. 1.1'} and providing both email and 
Xregular mail addresses.   ASCII versions of the code can also be mailed, if 
Xnecessary,  on a 3 1/2-inch double-sided high-density MS-DOS diskette, or a
X1/2 inch 9-track tape (1600 BPI, written on a UNIX machine).  For any 
Xquestions regarding the package, contact:
X
X\begin{center}
X\begin{verbatim}
X                             Prof. Eric Feigelson 
X                   Department of Astronomy & Astrophysics
X                        Pennsylvania State University 
X                          University Park PA  16802
X
X                          Telephone: (814) 865-0162 
X                               Telex: 842-510  
X                          Email: edf@astro.psu.edu
X\end{verbatim}
X\end{center}
X
XThe package consists of 59 subroutines of Fortran totally about 1/2 MBy. It
Xis completely self-contained, requiring no external libraries or programs.  The
Xcode is distributed in ten files: a brief READ.ME file; six files 
Xcalled asurv1.f-asurv6.f containing the source code; two documentation files,
Xwith the users manual in ASCII and LaTeX; and one file called asurv.etc
Xcontaining test datasets, test outputs and a subroutine list. We request 
Xthat  recipients not 
Xdistribute the package themselves beyond their own institution.  Rev. 0.0
Xof {\sl\bf ASURV} has been incorporated into the larger astronomical software 
Xsystem SDAS/IRAF, distributed by the Space Telescope Science Institute. 
X 
X     Installation requires removing the email headers, Fortran 
Xcompilation, linking, and executing.  We have written the code
Xconsistent with Fortran 77 conventions.  It has been successfully ported
Xto a Sun SPARCstation under UNIX, a DEC VAX under VMS, a personal computer
Xunder MS-DOS using Microsoft FORTRAN, and an IBM mainframe under VM/CMS.
XIt can also be ported to a Macintosh with small format changes (see \S
X{\bf A7} below).  
X\footnote{SPARCstation is a trademark of Sun Microsystems, Inc; UNIX is a 
Xtrademark of AT\& T Corporation; VAX and VMS are trademarks of Digital 
XEquipment Corporation; MS-DOS and Microsoft FORTRAN are trademarks of
XMicrosoft Corporation; VM/CMS is a trademark of International
XBusiness Machines Corporation; and Macintosh is a trademark of Apple
XCorporation.}
XWhen {\sl\bf ASURV} is compiled using Microsoft FORTRAN the
Xuser will notice several warnings from the compiler about labels used across
Xblocks and formal arguments not being used.  The user should not be alarmed,
Xthese warnings are caused by the compilation of the subroutines separately
Xand do not reflect program errors.
X
X     All of the functions have been tested against textbook formulae, 
Xpublished examples, and/or commercial software  packages.  However, some 
Xmethods are not widely used by researchers in other fields, and their behavior
Xis not well-documented.  We would appreciate hearing about any difficulties or
Xunusual behavior encountered when running the code. A bug report form for
X{\sl\bf ASURV} can be found in the asurv.etc file. 
X
X\bigskip
X\bigskip
X
X\noindent{\Large\bf A7  User Adjustable Parameters}
X
X       {\sl\bf ASURV} is initially set to handle data sets of up to 
X500 points, with up to  four variables, however a user may wish to 
Xconsider even larger data sets.  With this in mind, we have modified 
XRev 1.1 to be easy for a user to adjust to a given sample size.
X
X       The sizes of all the arrays in {\sl\bf ASURV} are controlled by 
Xtwo parameter statements; one is in {\sl\bf UNIVAR} and one is in 
X{\sl\bf BIVAR}.  Both statements are currently of the form:
X
X\begin{verbatim}
X                  PARAMETER(MVAR=4,NDAT=500,IBIN=50)
X\end{verbatim}  
X
XMVAR is the number of variables allowed in a data set 
Xand NDAT is the number of observations allowed in a data set.  To work 
Xwith larger data sets, it is only necessary to change the values MVAR and/or 
XNDAT in the two parameter statements.  If MVAR is set to be greater 
Xthan ten, then the data input formats should also be changed to read more 
Xthan ten variables from the data file.  Clearly, adjusting either MVAR or NDAT
Xupwards will increase the memory space required to run {\sl\bf ASURV}.
X
X       The following table provides listings of format statements that a user 
Xmight wish to change if data values do not match the default formats:  
X
X\begin{verbatim}
XInput Formats:
X
XProblem          Subroutine    Default Format      
X-------          ----------    --------------
XKaplan-Meier     DATAIN        Statement    30   FORMAT(10(I4,F10.3))
XTwo Sample tests 
X                 DATAIN                     40   FORMAT(I4,10(I4,F10.3))
XCorrelation/Regression
X                 DATREG                     20   FORMAT(I4,11F10.3)
X
X
X
X
XOutput Formats:
X
XProblem          Subroutine     Default Format
X-------          ----------     --------------
XKaplan-Meier     KMPRNT         Statements  550 [For Uncensored Points] 
X                                            555 [For Censored Points]  
X                                            850 [For Percentiles]
X                                           1000 [For Mean]
X                 KMDIF                      680 [For Differential KM]
X
XTwo Sample Tests TWOST          Statements 612, 660, 665,780
X        
X
XProblem          Subroutine     Default Format
X-------          ----------     --------------
XCorrelation/Regression
X Cox Regression  COXREG         Statements 1600, 1651, 1650
X Kendall's Tau   BHK                       2005, 2007
X Spearman's Rho  SPRMAN                     230, 240, 250, 997
X EM Algorithm    EM                        1150, 1200, 1250, 1300, 1350
X Buckley-James   BJ                        1200, 1250, 1300, 1350
X Schmitt         TWOKM                     1710, 1715, 1720, 1790, 1795, 1800,
X                                                 1805, 1820
X 
X\end{verbatim}                             
X
X     Macintosh users who have Microsoft Fortran may have difficulty with the
Xstatements that read from the keyboard and write to the screen.  Throughout
X{\sl\bf ASURV} we have used statements of the form,
X\begin{verbatim}
X           WRITE(6,format)        READ(5,format).
X\end{verbatim}
XMacintosh users may wish to replace the 6 and 5 in statements of the above
Xtype by `*'.
X           
X     VAX/VMS users may notice that the first character of some statements
Xsent to the screen is not printed.  This can be corrected by removing the 'C' 
Xat the start of the following statement, in the {\sl\bf ASURV} subroutine,
X\begin{verbatim}
X     C      OPEN(6,CARRIAGECONTROL='LIST',STATUS='OLD')
X\end{verbatim}
X
X     Finally, if the data have values which extend beyond three decimal places,
Xthen the user should reduce the value of `CONST' in the subroutine 
X{\sl\bf CENS} to have at least two more decimal places than the data.
X
X
X\newpage
X\noindent{\Large\bf A8  List of subroutines}
X
X     The following is an alphabetic listing of all the subroutines 
Xused in {\sl\bf ASURV} and their respective byte lengths.
X
X
X\bigskip
X\bigskip
X\centerline{\large\bf Subroutine List}
X
X\bigskip
X\begin{center}
X\begin{tabular}{|ll|ll|ll|} \hline
X   Subroutine & Bytes & Subroutine & Bytes & Subroutine & Bytes  \\ \hline
X&&&&& \\
X                AARRAY &4863 & FACTOR &1207 & REARRN & 3057 \\
X                AGAUSS &1345 & GAMMA &1520 & REGRES & 5505 \\
X                AKRANK &5887 & GEHAN &3477 & RMILLS & 1448 \\
X                ARISK &2675 & GRDPRB &11995 & SCHMIT & 8258\\
X                ASURV &7448 & KMADJ &3758 & SORT1 & 2512\\
X                BHK &6936 & KMDIF &5058 & SORT2 & 2214\\
X                BIN &15950 & KMESTM &7570 & SPEARHO & 2021\\
X                BIVAR &28072 & KMPRNT &8363 & SPRMAN & 8743\\
X                BJ &5750 & LRANK &6237 & STAT & 1859\\
X                BUCKLY &9830 & MATINV &3495 & SUMRY & 2324\\
X                CENS &2225 & MULVAR &15444 & SYMINV & 3007\\
X                CHOL &2672 & PCHISQ &2257 & TIE & 2784\\
X                COEFF &1948 & PETO &8054 & TWOKM & 16447\\
X                COXREG &8032 & PLESTM &6122 & TWOST & 15877\\
X                DATA1 &2650 & PWLCXN &2158 & UNIVAR & 27677\\
X                DATA2 &3470 & R3 &1629 & UNPACK & 1702\\
X                DATAIN &2962 & R4 &6142 & WLCXN & 6358\\
X                DATREG &4102 & R5 &3058 & XDATA & 1826\\
X                EM &12581 & R6 &9654 & XVAR & 5319\\
X                EMALGO &13183 & RAN1 &1432 & &\\ \hline
X\end{tabular}
X\end{center}
X\end{document}